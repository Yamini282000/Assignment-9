\documentclass[journal,12pt,twocolumn]{IEEEtran}
%
\usepackage{setspace}
\usepackage{gensymb}
\singlespacing
\usepackage[cmex10]{amsmath}
\usepackage{siunitx}
\usepackage{amsthm}

\usepackage{mathrsfs}

\usepackage{txfonts}
\usepackage{stfloats}

\usepackage{steinmetz}
\usepackage{cite}
\usepackage{cases}
\usepackage{subfig}
\usepackage{longtable}
\usepackage{multirow}
\usepackage{enumitem}
\usepackage{mathtools}
\usepackage{tikz}
\usepackage{circuitikz}
\usepackage{verbatim}
\usepackage{tfrupee}
\usepackage[breaklinks=true]{hyperref}
\usepackage{tkz-euclide} % loads  TikZ and tkz-base
\usetikzlibrary{calc,math}
\usetikzlibrary{fadings}
\usepackage{listings}
    \usepackage{color}                                            %%
    \usepackage{array}                                            %%
    \usepackage{longtable}                                        %%
    \usepackage{calc}                                             %%
    \usepackage{multirow}                                         %%
    \usepackage{hhline}                                           %%
    \usepackage{ifthen}                                           %%
  %optionally (for landscape tables embedded in another document): %%
    \usepackage{lscape}     
\usepackage{multicol}
\usepackage{chngcntr}
\DeclareMathOperator*{\Res}{Res}

\renewcommand\thesection{\arabic{section}}
\renewcommand\thesubsection{\thesection.\arabic{subsection}}
\renewcommand\thesubsubsection{\thesubsection.\arabic{subsubsection}}

\renewcommand\thesectiondis{\arabic{section}}
\renewcommand\thesubsectiondis{\thesectiondis.\arabic{subsection}}
\renewcommand\thesubsubsectiondis{\thesubsectiondis.\arabic{subsubsection}}

\hyphenation{op-tical net-works semi-conduc-tor}
\def\inputGnumericTable{}                                 %%

\lstset{
%language=C,
frame=single, 
breaklines=true,
columns=fullflexible
}
\begin{document}
%


\newtheorem{theorem}{Theorem}[section]
\newtheorem{problem}{Problem}
\newtheorem{proposition}{Proposition}[section]
\newtheorem{lemma}{Lemma}[section]
\newtheorem{corollary}[theorem]{Corollary}
\newtheorem{example}{Example}[section]
\newtheorem{definition}[problem]{Definition}
\newcommand{\BEQA}{\begin{eqnarray}}
\newcommand{\EEQA}{\end{eqnarray}}
\newcommand{\define}{\stackrel{\triangle}{=}}
\bibliographystyle{IEEEtran}
\providecommand{\mbf}{\mathbf}
\providecommand{\pr}[1]{\ensuremath{\Pr\left(#1\right)}}
\providecommand{\qfunc}[1]{\ensuremath{Q\left(#1\right)}}
\providecommand{\sbrak}[1]{\ensuremath{{}\left[#1\right]}}
\providecommand{\lsbrak}[1]{\ensuremath{{}\left[#1\right.}}
\providecommand{\rsbrak}[1]{\ensuremath{{}\left.#1\right]}}
\providecommand{\brak}[1]{\ensuremath{\left(#1\right)}}
\providecommand{\lbrak}[1]{\ensuremath{\left(#1\right.}}
\providecommand{\rbrak}[1]{\ensuremath{\left.#1\right)}}
\providecommand{\cbrak}[1]{\ensuremath{\left\{#1\right\}}}
\providecommand{\lcbrak}[1]{\ensuremath{\left\{#1\right.}}
\providecommand{\rcbrak}[1]{\ensuremath{\left.#1\right\}}}
\theoremstyle{remark}
\newtheorem{rem}{Remark}
\newcommand{\sgn}{\mathop{\mathrm{sgn}}}
\providecommand{\abs}[1]{\left\vert#1\right\vert}
\providecommand{\abs}[1]{\lvert#1\rvert} 
\providecommand{\res}[1]{\Res\displaylimits_{#1}} 
\providecommand{\norm}[1]{\left\lVert#1\right\rVert}
%\providecommand{\norm}[1]{\lVert#1\rVert}
\providecommand{\mtx}[1]{\mathbf{#1}}
\providecommand{\mean}[1]{E\left[ #1 \right]}
\providecommand{\fourier}{\overset{\mathcal{F}}{ \rightleftharpoons}}
%\providecommand{\hilbert}{\overset{\mathcal{H}}{ \rightleftharpoons}}
\providecommand{\system}{\overset{\mathcal{H}}{ \longleftrightarrow}}
	%\newcommand{\solution}[2]{\textbf{Solution:}{#1}}
\newcommand{\solution}{\noindent \textbf{Solution: }}
\newcommand{\cosec}{\,\text{cosec}\,}
\providecommand{\dec}[2]{\ensuremath{\overset{#1}{\underset{#2}{\gtrless}}}}
\newcommand{\myvec}[1]{\ensuremath{\begin{pmatrix}#1\end{pmatrix}}}
\newcommand{\mydet}[1]{\ensuremath{\begin{vmatrix}#1\end{vmatrix}}}
\numberwithin{equation}{subsection}
\makeatletter
\@addtoreset{figure}{problem}
\makeatother
\let\StandardTheFigure\thefigure
\let\vec\mathbf
\renewcommand{\thefigure}{\theproblem}
\def\putbox#1#2#3{\makebox[0in][l]{\makebox[#1][l]{}\raisebox{\baselineskip}[0in][0in]{\raisebox{#2}[0in][0in]{#3}}}}
     \def\rightbox#1{\makebox[0in][r]{#1}}
     \def\centbox#1{\makebox[0in]{#1}}
     \def\topbox#1{\raisebox{-\baselineskip}[0in][0in]{#1}}
     \def\midbox#1{\raisebox{-0.5\baselineskip}[0in][0in]{#1}}
\vspace{3cm}
\title{ASSIGNMENT-9}
\author{R.YAMINI}
\maketitle
\newpage
\bigskip
\renewcommand{\thefigure}{\theenumi}
\renewcommand{\thetable}{\theenumi}


%
\section{QUESTION No-2.2(Vectors)}
\item Let $\vec{A} = \myvec{4 \\ 2}$, $\vec{B}=\myvec{6 \\ 5}$ and $\vec{C}=\myvec{1 \\ 4}$ be the vertices of $\triangle ABC$.
\begin{description}
    \item [(a)] The median from $\vec{A}$ meets $BC$ at $\vec{D}$. Find the coordinates of the point $\vec{D}$.
    \item [(b)] Find the coordinates of the point $\vec{P}$ on $AD$
such that $AP : PD = 2 : 1$.
    \item [(c)] Find the coordinates of the points $\vec{Q}$ and $\vec{R}$
on medians $BE$ and $CF$ respectively such that $BQ : QE = 2 : 1$ and $CR : RF = 2 : 1$.
\end{description}

\section{Solution}
\begin{description}
\item [(a)] Given $\triangle ABC$ with vertices
\begin{align}
  \vec{A} = \myvec{4 \\ 2}, \vec{B}=\myvec{6 \\ 5},         \vec{C}=\myvec{1 \\ 4}  
\end{align}
Given that the median from $\vec{A}$ meets $BC$ at $\vec{D}$, now the coordinate of $\vec{D}$ is given as,
\begin{align}
    \vec{D} = \frac{\vec{B}+\vec{C}}{2} = \frac{\myvec{6 \\ 5}+\myvec{1 \\ 4}}{2}\\
    \implies \vec{D} = \myvec{\frac{7}{2} \\ \frac{9}{2}}
\end{align}
\end{description}
\begin{description}
\item [(b)] Result :The coordinates of point $\vec{C}$ dividing the line $AB$ in the ratio $m:n$ is given by 
\begin{align}
  \frac{n\vec{A}+m\vec{B}}{m+n}  \label{eq:1}
\end{align}
Given that the point $\vec{P}$ divides $AD$ in the ratio $2:1$, now to find $\vec{P}$ we use \eqref{eq:1},
\begin{align}
    \vec{P}=\frac{1\myvec{4\\2}+2\myvec{\frac{7}{2}\\ \frac{9}{2}}}{3}=\myvec{\frac{11}{3} \\ \frac{11}{3}}
\end{align}
\end{description}
\begin{description}
\item [(c)] Given that the point $\vec{Q}$ on the median $BE$ divides it in the ratio $2:1$, first we find $\vec{E}$,
\begin{align}
    \vec{E} =\frac{\vec{A}+\vec{C}}{2} = \frac{\myvec{4 \\ 2}+\myvec{1 \\ 4}}{2}\\
    \implies \vec{E} = \myvec{\frac{5}{2} \\ 3}.
\end{align}
Now we find $\vec{Q}$ using \eqref{eq:1}
\begin{align}
   \vec{Q}=\frac{1\myvec{6\\5}+2\myvec{\frac{5}{2}\\3}}{3}=\myvec{\frac{11}{3} \\ \frac{11}{3}} 
\end{align}
Similarly,Given that the point $\vec{R}$ on the median $CF$ divides it in the ratio $2:1$, first we find $\vec{F}$,
\begin{align}
    \vec{F} =\frac{\vec{A}+\vec{B}}{2} = \frac{\myvec{4 \\ 2}+\myvec{6 \\ 5}}{2}\\
    \implies \vec{F} = \myvec{5 \\ \frac{7}{2}}.
\end{align}
Now we find $\vec{R}$ using \eqref{eq:1}
\begin{align}
   \vec{R}=\frac{1\myvec{1\\4}+2\myvec{5 \\ \frac{7}{2}}}{3}=\myvec{\frac{11}{3} \\ \frac{11}{3}} 
\end{align}
\end{description}
The plot of the $\triangle ABC$ is given below
\numberwithin{figure}{section}
\begin{figure}[ht]
    \centering
    \includegraphics[width=\columnwidth]{TriangleABC.PNG}
    \caption{Plot of $\triangle ABC$}
    \label{fig:$\triangle ABC$}
\end{figure}


\end{document}